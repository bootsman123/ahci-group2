\documentclass[a4paper,11pt]{report}

% Math packages.
\usepackage{amssymb}
\usepackage{amstext}
\usepackage{amsmath}

% Monospace font.
\usepackage{inconsolata}

\begin{document}
	%%%%%%%%%%%%%%%%%%%%%%%%%%%%%%%%%%%%%%%%%%%%%%%%%%%%%%%%%%%%%%%%%%%%%%%%%%%
	\section{GOMSL Specification}
	In this section two GOSML specifications will be described. Namely one using a mobile phone as tangible object and another using touch gestures. Furthermore, the cooperative aspect of the game will not be specified; a single human playing the game is assumed.
	
		%%%%%%%%%%%%%%%%%%%%%%%%%%%%%%%%%%%%%%%%%%%%%%%%%%%%%%%%%%%%%%%%%%%%%%%
		\subsection{Custom Manual Operators}
		Besides the original manual operators several extra operators have to be defined to allow for the use of tangible objects and touch gestures.
		
		\subsection{}

# Manual operators (page 17).
Seve

Move_to target_object
Similar to Point_to, except that instead the mouse cursor is moved, the hand is moved towards the target object.

Grab target_object


// Tasks.
Task_item: T1
	Name is ....
	Type is ....
	Next is T2
	
Task_item: T2
	Name is ...
	Type is ...
	Next is None

// Specification using a mobile phone.
Method_for_goal: place object using <object_name>, and <location>
	Step 1. Accomplish_goal: // Grab mobile phone.
	Step 2. Accomplish_goal: // Select object <object_name>.
	Step 3. Accomplish_goal: // Place mobile phone <location>.
	Step 4. Return_with_goal_accomplished.
	
Method_for_goal: use object using <object_name>
	Step 1. Accomplish_goal: determine location
	Step 2. Accomplish_goal: place object using <object_name>, 


Method_for_goal: determine object
	Step 1. Think_of "Which object to use and where to place"
	Step 2. Look_for_object_whose Label is "mobile phone"
			and_store_under <location>.


Method_for_goal: determine location
	Step 1. Think_of "Which object to use and where to place".
	Step 2. Look_for_object_whose Label is ...
	 		and_store_under <location>.
	
	"(x, y)" under <location>.
	


Method_for_goal: use object using <object_name>, and <location>
	Step 1. Accomplish_goal: // Determine location.
	Step 2. Accomplish_goal: place object using "whistle", ...
	Step 3. Accomplish_goal: use object using "whistle"
	Step 4. Return_with_goal_accomplished.

// Specification using touch events.

	%%%%%%%%%%%%%%%%%%%%%%%%%%%%%%%%%%%%%%%%%%%%%%%%%%%%%%%%%%%%%%%%%%%%%%%%%%%
	\section{Mobile Phone and Server Communication}
	In order for the mobile phones and server to communicate with each other we define an protocol to which they should conform. In basis the mobile phone sends specific requests to the server and retrieves data in a specific format, namely JSON in this case.
	
	\paragraph{Connect}
	\texttt{SERVER_ADDRESS?action=connect&markId=_ID_}
	
	

\end{document}